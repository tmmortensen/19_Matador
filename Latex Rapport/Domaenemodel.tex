\section{Domænemodel}
Vi har i denne omgang valgt at modificere på vores \textbf{Domæne Model} fra \cite{19del2}, da denne allerede fungerede ret godt. Vi har dog måttet tilføje de nye domæner, som IOOuterActive har leveret i form af deres vision.


Vi kan se at det eneste ekstra domæne vi har tilføjet er \textit{Field}. Vi kunne have valgt at lave domæne klasser til vores forskellige felttyper også, men på den måde ville man hurtigt miste overblikket. Vi vurderede at det vigtigste var at vores kunde også ville kunne forstå modellen. Herudover har vi flyttet lidt rund på relationerne mellem domænerne. Før var det \textit{Game}, der kendte til \textit{DieCup}, men i denne omgang valgte vi at sige, det var mere logisk at man brugte raflebægeret på spillebrættet, hvilket nok er mere tro mod virkeligheden. Dette ses i figur \vref{fig:domain}

\begin{figure}[!ht]
    \centering
    \includegraphics[width=1\textwidth]{Domainmodel.pdf}
    \caption[<Text for the list of figures>]{Domæne Model}
    \label{fig:domain}
\end{figure}
\newpage